% Options for packages loaded elsewhere
\PassOptionsToPackage{unicode}{hyperref}
\PassOptionsToPackage{hyphens}{url}
\PassOptionsToPackage{dvipsnames,svgnames,x11names}{xcolor}
%
\documentclass[
]{article}

\usepackage{amsmath,amssymb}
\usepackage{lmodern}
\usepackage{iftex}
\ifPDFTeX
  \usepackage[T1]{fontenc}
  \usepackage[utf8]{inputenc}
  \usepackage{textcomp} % provide euro and other symbols
\else % if luatex or xetex
  \usepackage{unicode-math}
  \defaultfontfeatures{Scale=MatchLowercase}
  \defaultfontfeatures[\rmfamily]{Ligatures=TeX,Scale=1}
\fi
% Use upquote if available, for straight quotes in verbatim environments
\IfFileExists{upquote.sty}{\usepackage{upquote}}{}
\IfFileExists{microtype.sty}{% use microtype if available
  \usepackage[]{microtype}
  \UseMicrotypeSet[protrusion]{basicmath} % disable protrusion for tt fonts
}{}
\usepackage{xcolor}
\usepackage[top=20mm,left=18mm,right=18mm,heightrounded]{geometry}
\setlength{\emergencystretch}{3em} % prevent overfull lines
\setcounter{secnumdepth}{5}
% Make \paragraph and \subparagraph free-standing
\ifx\paragraph\undefined\else
  \let\oldparagraph\paragraph
  \renewcommand{\paragraph}[1]{\oldparagraph{#1}\mbox{}}
\fi
\ifx\subparagraph\undefined\else
  \let\oldsubparagraph\subparagraph
  \renewcommand{\subparagraph}[1]{\oldsubparagraph{#1}\mbox{}}
\fi


\providecommand{\tightlist}{%
  \setlength{\itemsep}{0pt}\setlength{\parskip}{0pt}}\usepackage{longtable,booktabs,array}
\usepackage{calc} % for calculating minipage widths
% Correct order of tables after \paragraph or \subparagraph
\usepackage{etoolbox}
\makeatletter
\patchcmd\longtable{\par}{\if@noskipsec\mbox{}\fi\par}{}{}
\makeatother
% Allow footnotes in longtable head/foot
\IfFileExists{footnotehyper.sty}{\usepackage{footnotehyper}}{\usepackage{footnote}}
\makesavenoteenv{longtable}
\usepackage{graphicx}
\makeatletter
\def\maxwidth{\ifdim\Gin@nat@width>\linewidth\linewidth\else\Gin@nat@width\fi}
\def\maxheight{\ifdim\Gin@nat@height>\textheight\textheight\else\Gin@nat@height\fi}
\makeatother
% Scale images if necessary, so that they will not overflow the page
% margins by default, and it is still possible to overwrite the defaults
% using explicit options in \includegraphics[width, height, ...]{}
\setkeys{Gin}{width=\maxwidth,height=\maxheight,keepaspectratio}
% Set default figure placement to htbp
\makeatletter
\def\fps@figure{htbp}
\makeatother
\newlength{\cslhangindent}
\setlength{\cslhangindent}{1.5em}
\newlength{\csllabelwidth}
\setlength{\csllabelwidth}{3em}
\newlength{\cslentryspacingunit} % times entry-spacing
\setlength{\cslentryspacingunit}{\parskip}
\newenvironment{CSLReferences}[2] % #1 hanging-ident, #2 entry spacing
 {% don't indent paragraphs
  \setlength{\parindent}{0pt}
  % turn on hanging indent if param 1 is 1
  \ifodd #1
  \let\oldpar\par
  \def\par{\hangindent=\cslhangindent\oldpar}
  \fi
  % set entry spacing
  \setlength{\parskip}{#2\cslentryspacingunit}
 }%
 {}
\usepackage{calc}
\newcommand{\CSLBlock}[1]{#1\hfill\break}
\newcommand{\CSLLeftMargin}[1]{\parbox[t]{\csllabelwidth}{#1}}
\newcommand{\CSLRightInline}[1]{\parbox[t]{\linewidth - \csllabelwidth}{#1}\break}
\newcommand{\CSLIndent}[1]{\hspace{\cslhangindent}#1}

\usepackage{float}
\makeatletter
\makeatother
\makeatletter
\makeatother
\makeatletter
\@ifpackageloaded{caption}{}{\usepackage{caption}}
\AtBeginDocument{%
\ifdefined\contentsname
  \renewcommand*\contentsname{Índice}
\else
  \newcommand\contentsname{Índice}
\fi
\ifdefined\listfigurename
  \renewcommand*\listfigurename{Lista de Figuras}
\else
  \newcommand\listfigurename{Lista de Figuras}
\fi
\ifdefined\listtablename
  \renewcommand*\listtablename{Lista de Tabelas}
\else
  \newcommand\listtablename{Lista de Tabelas}
\fi
\ifdefined\figurename
  \renewcommand*\figurename{Figura}
\else
  \newcommand\figurename{Figura}
\fi
\ifdefined\tablename
  \renewcommand*\tablename{Tabela}
\else
  \newcommand\tablename{Tabela}
\fi
}
\@ifpackageloaded{float}{}{\usepackage{float}}
\floatstyle{ruled}
\@ifundefined{c@chapter}{\newfloat{codelisting}{h}{lop}}{\newfloat{codelisting}{h}{lop}[chapter]}
\floatname{codelisting}{Listagem}
\newcommand*\listoflistings{\listof{codelisting}{Lista de Listagens}}
\makeatother
\makeatletter
\@ifpackageloaded{caption}{}{\usepackage{caption}}
\@ifpackageloaded{subcaption}{}{\usepackage{subcaption}}
\makeatother
\makeatletter
\@ifpackageloaded{tcolorbox}{}{\usepackage[many]{tcolorbox}}
\makeatother
\makeatletter
\@ifundefined{shadecolor}{\definecolor{shadecolor}{rgb}{.97, .97, .97}}
\makeatother
\makeatletter
\makeatother
\ifLuaTeX
\usepackage[bidi=basic]{babel}
\else
\usepackage[bidi=default]{babel}
\fi
\babelprovide[main,import]{portuguese}
% get rid of language-specific shorthands (see #6817):
\let\LanguageShortHands\languageshorthands
\def\languageshorthands#1{}
\ifLuaTeX
  \usepackage{selnolig}  % disable illegal ligatures
\fi
\IfFileExists{bookmark.sty}{\usepackage{bookmark}}{\usepackage{hyperref}}
\IfFileExists{xurl.sty}{\usepackage{xurl}}{} % add URL line breaks if available
\urlstyle{same} % disable monospaced font for URLs
\hypersetup{
  pdftitle={Distribuição Kumaraswamy},
  pdfauthor={Alisson Rosa   João Inácio   Vítor Pereira},
  pdflang={pt},
  colorlinks=true,
  linkcolor={blue},
  filecolor={Maroon},
  citecolor={Blue},
  urlcolor={Blue},
  pdfcreator={LaTeX via pandoc}}

\title{Distribuição Kumaraswamy}
\usepackage{etoolbox}
\makeatletter
\providecommand{\subtitle}[1]{% add subtitle to \maketitle
  \apptocmd{\@title}{\par {\large #1 \par}}{}{}
}
\makeatother
\subtitle{E suas Aplicações}
\author{Alisson Rosa João Inácio Vítor Pereira}
\date{}

\begin{document}
\maketitle
\begin{abstract}
Muitas vezes estamos interessados em modelar variáveis que estão
definidas entre zero e um, como sabemos aonde nossa variável esta
definida mas não sabemos qual dos valores será observado, temos portanto
uma incerteza probabilística, que pode e deve ser modelada por medidas
de probabilidade. Aqui portanto, introduziremos a distribuição
Kumaraswamy para o ajuste dos dados de desflorestamento, que é uma das
muitas possibilidades para modelagem desse tipo de variável,
encontraremos estimativas para os parâmetros da distribuição usando
estimadores de máxima verossimilhança para verificar a qualidade do
modelo realizamos comparações com a Distribuição Beta e a Distribuição
Normal.
\end{abstract}
\ifdefined\Shaded\renewenvironment{Shaded}{\begin{tcolorbox}[sharp corners, interior hidden, enhanced, frame hidden, boxrule=0pt, borderline west={3pt}{0pt}{shadecolor}, breakable]}{\end{tcolorbox}}\fi

\section{\centering Introdução}

Atualmente muitos fenômenos podem ser descritos como variáveis
aleatórias (va) definidas no intervalo unitário (0,1) \footnote{Onde
  parenteses indica limites do intervalo abertos.}, assim é natural que
pesquisadores desenvolvam distribuições de probabilidade que abarcam
esse tipo de va. Uma dessas distribuições é a Kumaraswamy, que foi
introduzida em Kumaraswamy (1980) como uma alternativa ao modelo beta
para aplicações na área de hidrologia. Em virtude deste fato, grande
parte dos trabalhos empíricos desta distribuição concentra-se nessa área
Nadarajah (2008). O presente trabalho visa contribuir na expansão e
utilização da Kumaraswamy, empregando modelos incondicionais para a taxa
de desfloresmento em diversos munícipios da Amazônia legal,
disponilizados
\href{http://www.dpi.inpe.br/prodesdigital/prodesmunicipal.php}{aqui}
pelo projeto PRODES do INPE. Assim é possível mensurar a qualidade da
Kumaraswamy para modelagem dos dados propostos, para isso estamos
utilizando 6 métricas frequentistas estabelecidas: AIC, BIC, CAIC,
Kolmogorov-Smirnov, Cramer-Von Mises e Anderson-Darling, em contraste
com a Distribuição Normal e a Distribuição Beta.

\section{\centering A distribuição Kumaraswamy}

Vamos nessa seção introduzir quantidades básicas da distribuição
Kumaraswamy, sendo elas sua função densidade de probabilidade (pdf),
função de distribuição acumulada (cdf), função quantilica (qf), função
de verossimilhança (ll) e esperança (\textbf{E})

\subsection{Quantidade Básicas}

Seja X uma variável aleatória que segue uma distribuição Kumaraswamy,
então sua cdf é dada por:

\begin{align}
F(x;\alpha, \beta) = 1 - (1 - x^\alpha)^\beta,  \quad 0 < x< 1
\end{align}

Onde \(\alpha, \beta > 0\). Sua pdf então fica definida como:

\begin{align}
f(x;\alpha, \beta) = \dfrac{dF}{dx} =\alpha\beta x^{\alpha - 1}(1 - x^\alpha)^{\beta  - 1}, \quad 0 < x< 1
\end{align}

Sua qf, que é a função inversa da cdf, fica definida como:

\begin{align}
Q(u;\alpha, \beta) = \bigg(1 - (1 - u)^{1/\beta}\bigg)^{\dfrac{1}{\alpha}}, \quad 0<u<1
\end{align}

É FÁCIL ver que que a esperança da distribuição Kumaraswamy é dada por

\begin{align}
\text{E}(X) = \dfrac{\beta\Gamma\bigg(1 + \dfrac{1}{\alpha}\bigg)\Gamma(\beta)}{\Gamma\bigg(1 + \dfrac{1}{\alpha} + \beta\bigg)}
\end{align}

A função de verossimilhança é dada por:

\begin{align}
L(\alpha, \beta; x) = \prod_{i=1}^{n}f(x;\alpha, \beta) = \alpha^n \beta^n \prod_{i=1}^{n}x_i^{\alpha - 1}\prod_{i=1}^{n}(1-x_i^{\alpha})^{\beta-1}
\end{align}

Para verificar as possibilidades de utilizações da Kumaraswamy em
contextos práticos é necessário conhecimento de sua densidade, assim nas
Figura~\ref{fig-density1} e Figura~\ref{fig-density2}. As figuras
referem-se a Estimativa da Densidade Kernel ou \emph{Kernel Density
Estimation} (KDE), método não-paramétrico para estimação da função
densidade com suavização e a Densidade Teórica, utilizando a
implementação da função densidade de probabilidade desenvolvida na
classe \texttt{Kuma} do python dos próprios autores. Podemos observar a
flexibilidade da Densidade da Kumaraswamy em 3 casos distintos: * Caso
1: \(\alpha = 0.5\) e \(\beta = 0.5\); * Caso 2: \(\alpha = 2\) e
\(\beta = 5\); * Caso 3: \(\alpha = 1\) e \(\beta = 2\).

\begin{figure}[H]

{\centering \includegraphics{report_files/figure-pdf/fig-density1-output-1.pdf}

}

\caption{\label{fig-density1}Função densidade por KDE da Kumaraswamy
para alguns valores de parâmetros}

\end{figure}

\begin{figure}[H]

{\centering \includegraphics{report_files/figure-pdf/fig-density2-output-1.pdf}

}

\caption{\label{fig-density2}Função densidade da Kumaraswamy para alguns
valores de parâmetros}

\end{figure}

A Figura~\ref{fig-cumulative} demonstra a curva da função acumulada para
os casos supracitados.

\begin{figure}[H]

{\centering \includegraphics{report_files/figure-pdf/fig-cumulative-output-1.pdf}

}

\caption{\label{fig-cumulative}Função acumulada da Kumaraswamy para
alguns valores de parâmetros}

\end{figure}

\subsection{Justificativa}

O artigo de Kumaraswamy (1980) propõe e demonstra aplicações da
distribuição Kumaraswamy para variáveis aleatórias e processos
aleatórios derivados de processos hidrológicos. O artigo foi publicado
na \emph{Journal of Hydrology}, assim é perceptível que a distribuição
foi concebida para se adequar a dados hidrológicos. Temos como casos de
suas utilizações em precipitação diária, fluxo diário, reservatórios de
água e análise das ondas do oceano, entre outras.

Para Nadarajah (2008) a utilização da Kumaraswamy para o campo da
hidrologia, é consolidada. Sendo perceptível pelas inúmeras aplicações
em diversos artigos como: Sundar e Subbiah (1989), Fletcher e
Ponnambalam (2008) e Koutsoyiannis e Xanthopoulos (1989), além de se
sobressair em relação a distribuição beta, a distribuição padrão para
dados no (0,1), de acordo com Koutsoyiannis e Xanthopoulos (1989). Em
Dey, Mazucheli, e Nadarajah (2018) a Kumaraswamy é utilizada para a
quantidade de deslocamento de líquido metálico em duas máquinas
diferentes, sendo superior as Distribuições Gumbel e Fréchet.

É notório também a dissiminação recente do estudo da Kumaraswamy, em
expansão para distribuições mais complexas, como em Lemonte,
Barreto-Souza, e Cordeiro (2013), Mazucheli et al. (2020), Cribari-Neto
e Santos (2019) e Sagrillo, Guerra, e Bayer (2021) ou em suas aplicações
em regressão e séries temporais, visto em Pumi, Rauber, e Bayer (2020),
Mitnik e Baek (2013), Fábio M. Bayer, Cribari-Neto, e Santos (2021) e
Fábio Mariano Bayer, Bayer, e Pumi (2017).

Assim é factível a falta de ajustes da Distribuição Kumaraswamy em
outras áreas de estudo, visto que a distribuição tem dissiminação
acadêmica exponencial recentemente, com admiráveis contribuições da
UFSM, em artigos supracitados. Decidimos contribuir para um maior estudo
da distribuição em outra área ambiental, o desmatamento, pensando
especificamente na Amazônia brasileira.

O estudo e modelos com bons ajustes de desmatamento e desflorestamento,
impactam a vida no mundo todo, tanto humana quanto não humana. Bons
modelos conseguem prever e informar quais variáveis mais impactam nos
desmatamento, possibilitando verificar sua evolução conforme o tempo.
Sendo útil para construção de políticas públicas, visando tomadas de
decisão mais eficientes.

O desflorestamento é questão com importância ambiental, social,
econômica e até política, pois a floresta amazônica tem seu papel no
armazenamento de carbono, evitando o aquecimento global, na reciclagem
de água e na manutenção da biodiversidade. Além de fornecerem uma grande
variedade de produtos materiais e de sustento para as populações locais.
Mesmo com áreas com grandes partes preservadas, há impacto na
biodiversidade, pois a distribuição das espécies não é uniforme. Muitas
espécies têm áreas de ocorrência restritas a partes que já foram
reduzida a pequenos fragmentos.

\section{\centering Apresentação dos dados}

Nesta seção são apresentados as análises descritivas para as variáveis
em estudo, por meio de tabelas e gráficos de barras. Na
Tabela~\ref{tbl-descritiva} é possível visualizar a analise descritiva
para a variável proporção de desmatamento (prop), contendo 760
observações. A média de desmatamento na região em estudo foi de 37.51\%
aproximadamente, com desvio padrão de 33.76\% aproximadamente. Podemos
notar, que neste banco de dados acabou ocorrendo valores mínimos de 0 e
máximo superiores a 1. Como a Distribuição Kumaraswamy suporta valores
no intervalo aberto (0,1), adaptamos os dados para estarem dentro deste
intervalo, valores iguais a 0 foram substituidos por 0.001, e valores
iguais ou maiores que 1 foram substituidos por 0.999, assim mantendo a
estrutura da distribuição.

\hypertarget{tbl-descritiva}{}
\begin{longtable}[]{@{}lr@{}}
\caption{\label{tbl-descritiva}Análise descritiva para a proporção de
desmatamento.}\tabularnewline
\toprule()
& prop \\
\midrule()
\endfirsthead
\toprule()
& prop \\
\midrule()
\endhead
count & 760 \\
mean & 0.375088 \\
std & 0.337627 \\
min & 0 \\
25\% & 0.0268267 \\
50\% & 0.317664 \\
75\% & 0.683185 \\
max & 1.00408 \\
\bottomrule()
\end{longtable}

Os estados da Amazônia Legal com maiores desmatamentos médios foram,
principalmente, em Rondônia com 0.60\% , Maranhão com 0.57\% e PA com
0.48\%. Na qual o estado de Rondônia apresenta 52 municípios, ocupando a
6ª posição de quantidade de municípios. Maranhão apresenta a maior
quantidade de municípios dentre os outros estados da Amazônia Legal, com
170 municípios. Podendo ser observado pelas Figura~\ref{fig-num-city} e
Figura~\ref{fig-states-mean}.

\begin{figure}[H]

{\centering \includegraphics{report_files/figure-pdf/fig-num-city-output-1.pdf}

}

\caption{\label{fig-num-city}Número de municípios em cada estado da
Amazônia Legal.}

\end{figure}

\begin{figure}[H]

{\centering \includegraphics{report_files/figure-pdf/fig-states-mean-output-1.pdf}

}

\caption{\label{fig-states-mean}Proporção média de desmatamento dos
Estados.}

\end{figure}

O estado de Maranhão detém o maior número de proporção de desmatamento
dentre os estados da Amazônia Legal, contendo municípios com proporções
máxima de desmatamento (100\%), sendo as cidades de Lago dos Rodrigues,
São Roberto, Igarepé Grande, Bom Lugar, Lago do Junco, Paulo Ramos,
Altamira do Maranhão, Olho d'Áqua das Cunhãs, Brejo de Areia, Lago da
Pedra, Presidente Médici. Estes dados podemos visualizar pelas
Figura~\ref{fig-states-max} e Figura~\ref{fig-city}. As taxas mínimas de
desmatamento, vistas pela Figura~\ref{fig-states-min}, se aproximaram em
praticamente todas as regiões. A região de Rondônia obteve a maior taxa
dentre as proporções mínimas de desmatamento, com 0.06\%, relativamente
muito próximo de 0\%, como visto em outros estados.

\begin{figure}[H]

{\centering \includegraphics{report_files/figure-pdf/fig-states-max-output-1.pdf}

}

\caption{\label{fig-states-max}Proporção máxima de desflorestamento dos
Estados.}

\end{figure}

\begin{figure}[H]

{\centering \includegraphics{report_files/figure-pdf/fig-states-min-output-1.pdf}

}

\caption{\label{fig-states-min}Proporção mínima de desmatamento dos
Estados.}

\end{figure}

\begin{figure}[H]

{\centering \includegraphics{report_files/figure-pdf/fig-city-output-1.pdf}

}

\caption{\label{fig-city}Cidades com maiores desflorestamento na
Amazônia legal.}

\end{figure}

\section{Ajuste Inicial}

A construção e avaliação numérica dos Estimadores de Máxima
Verossimilhança (EMV) foi realizado via implementação da
log-verossimilhança negativa, assim a otimização computacional com a
função \texttt{minimize} e o método Nelder-Mead, ambos da biblioteca
\texttt{scipy} e assim, como toda a construção do presente trabalho foi
desenvolvida em python.

Assim utilizamos como chute inicial \(\alpha = 0.5\) e \(\beta = 1\), e
como é possível observar nos ajustes de densidade das
Figura~\ref{fig-kde} e Figura~\ref{fig-theoric-density}, foi obtido um
ajuste razoável, se ajustando de maneira coerente ao histrograma,
ficando os EMVs \(\widehat{\alpha}\) =
\texttt{python\ criterios{[}\textquotesingle{}alpha\textquotesingle{}{]}}
e \(\widehat{\beta}\) =
\texttt{python\ criterios{[}\textquotesingle{}beta\textquotesingle{}{]}}.

\begin{figure}[H]

{\centering \includegraphics{report_files/figure-pdf/fig-kde-output-1.pdf}

}

\caption{\label{fig-kde}Densidade ajustada por KDE.}

\end{figure}

\begin{figure}[H]

{\centering \includegraphics{report_files/figure-pdf/fig-theoric-density-output-1.pdf}

}

\caption{\label{fig-theoric-density}Ajuste da densidade teórica.}

\end{figure}

\subsection{Medidas Básicas}

\section{\centering Ajuste do Modelo}

Para a verificação da adequação da distribuição Kumaraswamy para
modelagem de dados de desmatamento iremos realizar a comparação com duas
distribuições amplamente utilizadas: Distribuição Normal e a
Distribuição beta, a distribuição mais utilizada para variáveis
aleatórias com suporte no (0,1).

A comparação foi construída utilizando 6 métricas: AIC, BIC, CAIC,
Kolmogorov-Smirnov, Cramer-Von Mises e Anderson-Darling, que se baseiam
principalmente na verossimilhança. A verossimilhança a considera os
parâmetros variáveis, assim a função de verossimilhança indica os
parâmetros mais plausíveis de terem gerado a amostra. Logo, podemos
verificar entre todas as distribuições quais possuem as maiores
verossimilhanças, tendo assim os parâmetros mais plausíveis para a
geração da amostra. No entanto, nota-se que todos os critérios contam
outros elementos para a sua construção.

\hypertarget{tbl-planet-measures}{}
\begin{longtable}[]{@{}lrrrl@{}}
\caption{\label{tbl-planet-measures}Métricas para comparação da
Distribuição Kumaraswamy em contraste à Distribuição Normal e
Beta}\tabularnewline
\toprule()
& Kumaraswamy & Beta & Normal & Vencedor \\
\midrule()
\endfirsthead
\toprule()
& Kumaraswamy & Beta & Normal & Vencedor \\
\midrule()
\endhead
AIC & -440.934 & -443.086 & -154637 & Normal \\
BIC & -440.918 & -433.82 & -154628 & Normal \\
CAIC & -431.667 & -443.07 & -154637 & Normal \\
AD & 18.1063 & 9.73665 & 767.893 & Beta \\
CVM & 9.69198 & 1.25157 & 64.3781 & Beta \\
KS & 0.0887461 & 0.0891663 & 0.537076 & Kumaraswamy \\
\bottomrule()
\end{longtable}

É intrigante o exposto pela Tabela~\ref{tbl-planet-measures}, onde
podemos perceber que todas as distribuições ganham em alguma métrica, a
Distribuição Normal se notabiliza por ter ganhado em todas as métricas
que levam em consideração a verossimilhança diretamente. Enquanto as
Distribuições Unitárias evidenciaram-se em métricas mais adequadas para
Distribuições não encaixadas, assim é possível dizer que a Kumaraswamy é
uma possível competidora para a Distribuição Beta em modelos
incondicionais.

\section{\centering Conclusão}

Neste trabalho analisamos em especial a Distribuição Kumaraswamy, junto
com o banco de dados para a taxa de desfloresmento em diversos
munícipios da Amazônia Legal. De tal forma a ser comparada com outras
duas distribuições, a Distribuição Beta e a Distribuição Normal.

A Distribuição Kumaraswamy é uma grande concorrente da Distribuição
Beta, e por conta de possuir cdf de forma fechada, ao contrário da
distribuição Beta, tem uma maior relevância na utilização em questões de
simulação.

Desta forma, conseguimos perceber que a Distribuição Kumaraswamy
demostrou-se realmente uma grande concorrente da Distribuição Beta.
Porém, a Distribuição Beta ainda conseguiu se sair melhor, ganhando em
praticamente todas as métricas, quando as duas comparadas separadamente
(descartando a Normal), apenas no teste de KS a Distribuição Kumaraswamy
se saiu melhor. Portanto, a distribuição que melhor representou os dados
de desflorestamento da Amazônia Legal foi a Distribuição Beta.

\section{\centering Referências}

\hypertarget{refs}{}
\begin{CSLReferences}{1}{0}
\leavevmode\vadjust pre{\hypertarget{ref-bayer2017kumaraswamy}{}}%
Bayer, Fábio Mariano, Débora Missio Bayer, e Guilherme Pumi. 2017.
{"Kumaraswamy autoregressive moving average models for double bounded
environmental data"}. \emph{Journal of Hydrology} 555: 385--96.

\leavevmode\vadjust pre{\hypertarget{ref-bayer2021inflated}{}}%
Bayer, Fábio M, Francisco Cribari-Neto, e Jéssica Santos. 2021.
{"Inflated Kumaraswamy regressions with application to water supply and
sanitation in Brazil"}. \emph{Statistica Neerlandica} 75 (4): 453--81.

\leavevmode\vadjust pre{\hypertarget{ref-cribari2019inflated}{}}%
Cribari-Neto, Francisco, e Jessica Santos. 2019. {"Inflated Kumaraswamy
distributions"}. \emph{Anais da Academia Brasileira de Ci{ê}ncias} 91.

\leavevmode\vadjust pre{\hypertarget{ref-dey2018kumaraswamy}{}}%
Dey, Sanku, Josmar Mazucheli, e Saralees Nadarajah. 2018. {"Kumaraswamy
distribution: different methods of estimation"}. \emph{Computational and
Applied Mathematics} 37 (2): 2094--2111.

\leavevmode\vadjust pre{\hypertarget{ref-fletcher}{}}%
Fletcher, SG, e K Ponnambalam. 2008. {"Stochastic control of reservoir
systems using indicator functions: new enhancements"}. \emph{Water
Resources Research} 44 (12).

\leavevmode\vadjust pre{\hypertarget{ref-ganji2006grain}{}}%
Ganji, A, K Ponnambalam, D Khalili, e M Karamouz. 2006. {"Grain yield
reliability analysis with crop water demand uncertainty"}.
\emph{Stochastic Environmental Research and Risk Assessment} 20 (4):
259--77.

\leavevmode\vadjust pre{\hypertarget{ref-koutsoyiannis}{}}%
Koutsoyiannis, Demetris, e Themistocle Xanthopoulos. 1989. {"On the
parametric approach to unit hydrograph identification"}. \emph{Water
resources management} 3 (2): 107--28.

\leavevmode\vadjust pre{\hypertarget{ref-kuma}{}}%
Kumaraswamy, Ponnambalam. 1980. {"A generalized probability density
function for double-bounded random processes"}. \emph{Journal of
hydrology} 46 (1-2): 79--88.

\leavevmode\vadjust pre{\hypertarget{ref-lemonte2013exponentiated}{}}%
Lemonte, Artur J, Wagner Barreto-Souza, e Gauss M Cordeiro. 2013. {"The
exponentiated Kumaraswamy distribution and its log-transform"}.
\emph{Brazilian Journal of Probability and Statistics} 27 (1): 31--53.

\leavevmode\vadjust pre{\hypertarget{ref-mazucheli2020unit}{}}%
Mazucheli, J, AFB Menezes, LB Fernandes, RP De Oliveira, e ME Ghitany.
2020. {"The unit-Weibull distribution as an alternative to the
Kumaraswamy distribution for the modeling of quantiles conditional on
covariates"}. \emph{Journal of Applied Statistics} 47 (6): 954--74.

\leavevmode\vadjust pre{\hypertarget{ref-mitnik2013new}{}}%
Mitnik, Pablo A. 2013. {"New properties of the Kumaraswamy
distribution"}. \emph{Communications in Statistics-Theory and Methods}
42 (5): 741--55.

\leavevmode\vadjust pre{\hypertarget{ref-mitnik2013kumaraswamy}{}}%
Mitnik, Pablo A, e Sunyoung Baek. 2013. {"The Kumaraswamy distribution:
median-dispersion re-parameterizations for regression modeling and
simulation-based estimation"}. \emph{Statistical Papers} 54 (1):
177--92.

\leavevmode\vadjust pre{\hypertarget{ref-nadarajah2008distribution}{}}%
Nadarajah, Saralees. 2008. {"On the distribution of Kumaraswamy"}.
\emph{Journal of Hydrology} 348 (3): 568--69.

\leavevmode\vadjust pre{\hypertarget{ref-pumi2020kumaraswamy}{}}%
Pumi, Guilherme, Cristine Rauber, e Fábio M Bayer. 2020. {"Kumaraswamy
regression model with Aranda-Ordaz link function"}. \emph{Test} 29 (4):
1051--71.

\leavevmode\vadjust pre{\hypertarget{ref-r}{}}%
R Core Team. 2022. \emph{R: A Language and Environment for Statistical
Computing}. Vienna, Austria: R Foundation for Statistical Computing.
\url{https://www.R-project.org/}.

\leavevmode\vadjust pre{\hypertarget{ref-sagrillo2021modified}{}}%
Sagrillo, Murilo, Renata Rojas Guerra, e Fábio M Bayer. 2021. {"Modified
Kumaraswamy distributions for double bounded hydro-environmental data"}.
\emph{Journal of Hydrology} 603: 127021.

\leavevmode\vadjust pre{\hypertarget{ref-sundar}{}}%
Sundar, V, e K Subbiah. 1989. {"Application of double bounded
probability density function for analysis of ocean waves"}. \emph{Ocean
engineering} 16 (2): 193--200.

\end{CSLReferences}



\end{document}
